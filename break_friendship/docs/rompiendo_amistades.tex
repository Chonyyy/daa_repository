\documentclass{article}
\usepackage[utf8]{inputenc}
\usepackage[spanish]{babel}
\usepackage{amsmath}
\usepackage{amssymb}
\usepackage{hyperref}
\usepackage{graphicx}
\usepackage{csquotes}

\newtheorem{problema}{Problema}[section]

\title{\bf{Rompiendo amistades}}
\date{}
\author{María de Lourdes Choy Fernández \\ Alejandro Yero Valdéz \\ Leismael Sosa}

\begin{document}
\pagestyle{empty}
\maketitle
\thispagestyle{empty}
\begin{center}
\includegraphics[scale=1]{images}
\end{center}
\newpage

\tableofcontents

\newpage
\setcounter{page}{1}
\pagestyle{plain}

\section{Problema}
% Descripción detallada del problema a resolver. Incluir ejemplos y casos de prueba si es necesario.
Por algún motivo, a María no le gustaba la paz y le irritaba que sus compañeros de aula se llevaran tan bien.
Ella quería ver el mundo arder. Un día un demonio se le acercó y le propuso un trato: \enquote{A cambio de un cachito de tu alma,
te voy a dar el poder para romper relaciones de amistad de tus compañeros, con la única condición de que no puedes romperlas todas}.
Sin pensarlo mucho (Qué más da un pequeño trocito de alma), María aceptó y se puso a trabajar. Ella conocía,
dados sus k compañeros de aula, quiénes eran mutuamente amigos.

Como no podía eliminar todas las relaciones de amistad, pensó en qué era lo siguiente que podía hacer más daño.
Si una persona quedaba con uno o dos amigos, podría hacer proyectos en parejas o tríos (casi todos los de la
carrera son así), pero si tenía exactamente tres amigos, cuando llegara un proyecto de tres personas, uno de sus
amigos debería quedar afuera y se formaría el caos.

Ayude a María a saber si puede sembrar la discordia en su aula eliminando relaciones de amistad entre sus
compañeros de forma que todos queden, o bien sin amigos, o con exactamente tres amigos.

\subsection{Formalización del problema}

Este problema se puede describir en términos de teoría de grafos. Aquí el grafo sería no dirigido,
donde los compañeros de María serían los nodos del grafo y una arista entre 2 nodos representa la
amistad entre 2 compañeros.

Como María quiere que todos sus compañeros estén sin amigos o con exactamente 3 amigos, sin romper todas
las relaciones, entonces se puede decir que el objetivo de este problema es encontrar en $G$ un subgrafo $G'$
tal que todos sus nodos tengan grado 0 o 3 y donde exista al menos 1 de grado 3 (no puede romper todas las amistades).
Si ignoramos todos los nodos de grado 0, entonces el problema se puede reformular en el siguiente.

\begin{problema}\label{prob_1}
 Dado un grafo no dirigido $G = (V, E)$, el objetivo es determinar si existe en $G$ un subgrafo $G' = (V', E')$
 tal que todos los vértices en $G'$ tengan exactamente grado 3, es decir, tal que $G'$ sea regular de orden 3.
\end{problema}

\section{Demostración de NP completitud}

Para demostrar que el problema \ref{prob_1} es \textbf{NP-completo} primero demostraremos que es subconjunto de \textbf{NP}. Esto es sencillo
de ver pues si tenemos un grafo $G'$ de $G$, el verificar que $G'$ es un grafo regular de grado 3 (llamado en algunas bibliografías grafo cúbico)
se puede realizar en tiempo polinomial iterando por todos sus nodos y verificando que todos tengan grado 3.

Ahora, para demostrar que es \textbf{NP-completo} el problema de decisión, de determinar si un grafo $G$ tiene un subgrafo regular de grado 3 (de ahora en adelante lo llamaremos como cúbico),
tenemos que reducir algún problema \textbf{NP-completo} a nuestro problema. Esto lo haremos con el problema de $3$-coloración que fue demostrado
\textbf{NP-completo} por Karps, ver \cite{3-color}.

El problema de 3 coloración es el siguiente:

\begin{problema}\label{prob_2}\textbf{3-coloración}

    Dado un grafo no dirigido $G = (V,E)$, determinar si existe alguna partición $V_1,V_2,V_3$ de $V$ tal que se cumpla que
    para todo para de nodos $x,y\in V_i$, no existe la arista $<x,y>$
\end{problema}

En la siguiente sub-sección demostraremos la reducción del problema \ref{prob_2} al \ref{prob_1}

\subsection{Reducción}

Para realizar la reducción, tenemos que encontrar una forma de transformar un grafo $G$ en otro grafo $G'$, tal que
en $G$ exista una 3-coloración si y solo si en $G'$ existe un subgrafo cúbico.

Es decir, lo primero que tenemos que encontrar es una función $f(G) \to G'$ que traduzca un problema de 3-coloración en un
problema de determinar si existe un subgrafo cúbico. La forma en la que formaremos $G'$ dado $G$, es en esencia creando nuevos nodos
que formen ciclos y otros nuevos que se conecten de forma inteligente a estos ciclos. La intuición detrás de esta idea es debida
a la propiedad de que todos los nodos involucrados en un ciclo tienen grado 2 y cuando conectas nuevos nodos a estos ciclos
empiezas a tener nodos de grado 3 y si escoges cuidadosamente un subconjunto de todos los nodos generados es bastante probable que todos
estos tengan grado 3. Ahora, lo que haremos es mostrar precisamente como construir los ciclos y los nodos que se conectan a estos, de forma
que se encuentre una equivalencia entre los problemas en cuestión.


Sea $G = (V, E)$ un grafo de entrada al problema de 3-coloración. Denotemos sus nodos como
$v_1,v_2,\dots,v_n$ y sus aristas como $e_1,e_2,\dots,e_m$. También para todo nodo $v_i$ definimos
$d_i$ como su grado, y a $\delta_i$ como el conjunto de los indices de todas las aristas que inciden en $v_i$.
Ordenando el conjunto $\delta_i$ de menor a mayor, obtenemos $\delta_{i,1}, \delta_{i,2},\dots,\delta_{i,d_i}$,
donde nuevamente, $\delta_{i,j}$ representa el indice de la arista que ocupa la posición $j$ en el orden definido.

Ahora, para construir el grafo $G' = (V', E')$, comenzaremos definiendo cuales
son los nodos de este grafo y luego cuales son sus aristas.

Los nodos de $G'$ se construirán de la siguiente forma:
\begin{itemize}
    \item Por cada nodo $v_i\in V$ crearemos los nodos $v_{i,j}^h$ para cada $1\leq j \leq 3d_i + 2$ y $1\leq h\leq 3$. Que denominaremos \textit{nodos ciclo}.
    \item Por cada arista $e_i\in E$ crearemos los llamados \textit{nodos aristas} $a_j^h$ para cada $1\leq h \leq 3$.
    \item Por cada nodo $v_i\in V$ crearemos 2 nodos $q_i$ y $r_i$ que denominaremos \textit{nodos representantes}, pues solo existirá un par de estos por cada nodo de $V$.
\end{itemize}

Con la construcción de las aristas del nuevo grafo $G'$ se podrá apreciar cuales son los nodos que forman ciclos y cuales
son los que se conectan a estos ciclos. El conjunto de las aristas $E'$ de $G'$ se construye como sigue:
\begin{itemize}
    \item Para todo $1\leq i\leq n$, donde $n=|V|$, se crea una arista entre los nodos $v_{i,j}^h$ y $v_{i,j+1}^h$ (donde el subíndice $j$ se toma módulo $3d_i + 2$), para todo $1\leq h\leq 3$. Esto creará un ciclo entre los nodos $v_{i,j}^h$ (de aquí sale el nombre de \textit{nodos ciclos}) y se puede observar que se crean exactamente 3 ciclos por cada $v_i\in V$. Por comodidad denotaremos estos ciclos por $C_i^h $
    \item Para todo nodo $v_i\in V$ y para todo $1\leq h\leq 3$, se crearán aristas entre el \textit{nodo arista} $a_{\delta_{i,j}}^h$ y los 3 nodos consecutivos $v_{i,3j}^h, v_{i,3j+1}^h, v_{i,3j+2}^h$ del ciclo $C_i^h$ para todo $1\leq j \leq d_i$. Pero como el nodo $a_{\delta_{i,j}}^h$ representa una arista en $G$ entre 2 nodos, se tiene que el nodo $a_{\delta_{i,j}}^h$ estará conectado a otros 3 nodos consecutivos del ciclo $C_{\delta_{i,j}}^h$, para un total de 6 aristas conectadas a cada \textit{nodo arista}
    \item Para todo nodo $v_i\in V$ creamos una arista entre el \textit{nodo representante} $q_i$ y los nodos $v_{i,1}^h$ y $v_{i,2}^h$, para cada $1\leq h\leq 3$. Observar que estos son los nodos que no se conectaron a los \textit{nodos aristas}.
    \item Para todo nodo $r_i$ creamos una arista con el nodo $r_{i+1}$ (módulo $n$), formando otro ciclo en $G'$, a este ciclo lo denotaremos por $R$
    \item Por último, para todo $1\leq i\leq n$, creamos una arista entre los nodos $q_i$ y $r_i$.
\end{itemize}

Ahora que tenemos definida la transformación del grafo $G$ en el grafo $G'$, solo queda demostrar que en $G$ existe una 3-coloración si y solo si en $G'$ existe un subgrafo cúbico.
\newline

$(\Rightarrow)$ Demostración de la primera implicación.

Supongamos que $G = (V,E)$ es un grafo 3-colorable, esto quiere decir que los nodos de $V$ se pueden
particionar en 3 conjuntos $V^1,V^2,V^3$, donde los nodos que pertenecen al conjunto $V^i$ con $1\leq i\leq 3$
tienen el mismo color y ninguna arista entre ellos. Ahora, del grafo $G'$ generado por $G$, tenemos que encontrar un subgrafo cúbico.
Para esto tomaremos el grafo $G''$ como el subgrafo inducido en $G'$ por los nodos siguientes:
\begin{itemize}
    \item Para todo $1\leq i\leq n$, donde $n=|V|$ tomaremos los \textit{nodos representantes} $q_i$ y $r_i$.
    \item Tomaremos todos los \textit{nodos aristas} $a_i^h$, para $1\leq i\leq m$ y $1\leq h\leq 3$, donde $m=|E|$.
    \item Para todo nodo $v_i\in V^h$ tomaremos todos los nodos del ciclo $C_i^h$ para $1\leq j\leq 3$. Notar que por cada $v_i$ solo tomamos 1 de los 3 ciclos que genera en $G'$.
\end{itemize}

Ahora, solo tenemos que demostrar que $G''$ solo contiene nodos de grado 0 o 3, de donde si nos quedamos con el grafo resultante de
quitar de $G''$ todos los nodos de grado 0 tendremos la solución.

Los primeros nodos que se pueden observar de grado 3 son los nodos $r_i$, debido a que al formar el ciclo $R$ todos los nodos $r_i$ tienen grado 2 y al
estar conectados siempre con los nodos $q_i$ se tiene que todos los $r_i$ son de grado 3.

Los nodos $q_i$ al estar conectados con los nodos $r_i$ tienen grado al menos 1, pero se tiene que también están conectados con exactamente 1 par $v_{i,1}^h, v_{i,2}^h$ (de los 3 posibles pares), exactamente el par con $h$ igual
a $V^h$, donde $V^h$ es el conjunto al que pertenece el nodo $v_i$, debido a que en $G''$ por cada $v_i$ existe un solo ciclo $C_i^h$.

Sea $v_{x,y}^k$ un \textit{nodo ciclo} arbitrario en $G''$. Como para todo nodo $v_i\in V$ nos habíamos quedado con todos los nodos del ciclo $C_i^h$, entonces se tiene que si el
\textit{nodo ciclo} $v_{x,y}^k$ pertenece a $G''$ entonces esta conectado con sus 2 \textit{nodos ciclos} adyacentes en el ciclo $C_x^k$. Pero también por la forma de construcción de $G'$
se sabe que cada nodo $v_{x,y}^k$ también está conectado con exactamente 1 \textit{nodo representante} $q_x$ o un \textit{nodo arista} $a_{\delta_{x,y}}^k$ para algún $1\leq y\leq d_x$.
De donde se tiene que cada \textit{nodo ciclo} $v_{x,y}^k$ tiene grado 3 (no puede tener un mayor grado debido a que en $G'$ todos los \textit{nodos ciclos} tienen exactamente grado 3).

Para culminar el análisis tomemos un \textit{nodo arista} $a_i^h$ arbitrario, y supongamos que su arista asociada en el grafo $G$ es $e_i=<v_x,v_y>$. Aquí hay que observar
que por construcción el \textit{nodo arista} $a_i^h$ está conectado a los 3 \textit{nodos ciclos} consecutivos $v_{x,3j}^h, v_{x,3j+1}^h, v_{x,3j+2}^h$ para algún $1\leq j\leq d_x$ y a los 3
\textit{nodos ciclos} consecutivos $v_{y,3k}^h, v_{y,3k+1}^h, v_{y,3k+2}^h$ para algún $1\leq k\leq d_y$.
Pero como $v_x$ y $v_y$ son adyacentes, entonces en el grafo $G$ pertenecen a particiones distintas de $V^1,V^2,V^3$, por lo que a lo sumo uno de los ciclos $C_x^h$ y $C_y^h$ pertenecerá a $G''$.
De lo anterior se deduce que todos los \textit{nodos aristas} en $G''$ tienen grado 0 o 3.

Por todo lo demostrado anteriormente concluimos que $G'$ tiene un subgrafo cúbico.\newline

$(\Leftarrow)$ Demostración de la segunda implicación.

Ahora tenemos que demostrar que si $G'$ tiene un subgrafo cúbico $G''$, entonces el grafo $G$ es 3-colorable.
La idea de la demostración viene por observar que todo subgrafo cúbico $G''$ de $G'$ tiene que contener un subconjunto de los
ciclos $C_i^h$ de $G'$ y como se demostrará que en $G''$ solo puede existir 1 de los 3 ciclos $C_i^1, C_i^2,C_i^3$ generados por cada
nodo $v_i$ de $V$ y solamente 1 ciclo entre $C_i^h$ y $C_j^h$ si existe una arista en $G$ entre los nodos $v_i$ y $v_h$, pues se concluirá que colocando los nodos $v_i$, representantes de los ciclos $C_i^h$, en la partición $V^h$
se tiene una 3-coloración valida en $G$. Formalmente la demostración es como sigue:

Sea $V''$ el conjunto de los nodos de $G''$, entonces en $V''$ debe existir al menos un \textit{nodo ciclo} $v_{i,j}^h$, pues en caso contrario las únicas opciones
para formar el subgrafo cúbico son los \textit{nodos aristas} $a_i^h$, y los \textit{nodos representantes} $q_i$ y $r_i$. Pero con estos nodos no se puede formar un subgrafo cúbico.
En efecto, los \textit{nodos aristas} $a_i^h$ solo tienen conexiones con los \textit{nodos ciclos}, pero al no existir ninguno, los $a_i^h$ tienen grado 0.
El \textit{nodo representante} $q_i$ solo tiene conexiones con los \textit{nodos ciclos} que no están y con exactamente el \textit{nodo representante} $r_i$, de donde $q_i$ tendría grado 1
si los $r_i$ están presentes y grado 0 si no lo estuvieran, de donde los $q_i$ no pueden componer un subgrafo cúbico. Por otro lado los \textit{nodos representantes} $r_i$ al formar el ciclo $R$
tienen como mínimo grado 2 y obtendrían grado 3 si estuvieran presentes los nodos $q_i$, debido a que se conectan exactamente con el nodo $q_i$ en $G'$, pero sabemos que no pueden estar presentes en un subgrafo
cúbico si no existen \textit{nodos ciclos}.
Con esto se concluye que todo subgrafo cúbico de $G''$ tiene que contener al menos un \textit{nodo ciclo} $v_{i,j}^h$.

Como sabemos que $G''$ contiene al menos un \textit{nodo ciclo} $v_{i,j}^h$, ahora se demostrará que si $v_{i,j}^h\in V''$ entonces todos los nodos del ciclo $C_i^h$ deben estar en $V''$. Esto se ve
claramente en que los nodos $v_{i,j}^h$ además de estar conectados con sus adyacentes en $C_i^h$, por construcción, tienen exactamente una conexión con un \textit{nodo arista} o el \textit{nodo representante} $q_i$,
de donde para que el grafo $G''$ sea cúbico debe estar presente en $V''$ el nodo con el que $v_{i,j}^h$ está conectado fuera del ciclo además de sus \textit{nodos ciclos} adyacentes. Esto también trae como consecuencia
que si $v_{i,j}^h\in V''$, entonces todos los nodos del ciclo $C_i^h$ están en $V''$.

También tenemos que si el ciclo $C_i^k$ está presente en $G''$ entonces el \textit{nodo ciclo} $v_{i,1}^k\in V''$, pero para que su grado sea 3, tiene que ocurrir que el
\textit{nodo representante} $q_i$ esté conectado a $v_{i,1}$, pero si $q_i\in V''$, entonces tiene que ocurrir que el \textit{nodo representante} $r_i\in V''$ pues de otra forma $q_i$ tendría grado 2.
De esto se concluye que todos los \textit{nodos representantes} $q_i$ y $r_i$ pertenecen a $V''$. Pero también de aquí se demuestra que en $G''$ solo puede existir un ciclo entre
$C_i^1, C_i^2, C_i^3$ pues en caso contrario $q_i$ estaría conectado a 2 nodos de al menos 2 de estos ciclos, provocando que el grado de $q_i$ sea de mínimo 5 y esto evitaría que $q_i$ estuviera en $G''$
lo cual es absurdo pues demostramos que pertenece.

Como se sabe que en $G''$ se encuentran ciclos $C_i^k$, entonces se sabe que tienen que existir \textit{nodos aristas} de la forma $a_i^k$ pues en caso contrario existirían en $C_i^k$ \textit{nodos ciclos}
de la forma $v_{i, 3j}^k, v_{i, 3j + 1}^k, v_{i, 3j + 2}^k$ para algún $j$ que tendrían solamente grado 2 (por construcción). Entonces, supongamos que el \textit{nodo arista} $a_i^h$ está en $G''$,
de donde su grado es 3. Como $a_i^h$ es un nodo que en $G$ está asociado con una arista $e_i = <v_x, v_y>$ (por construcción), se tiene que $a_i^h$ está conectado a 3 nodos de solamente un ciclo entre $C_x^h$ y  $C_y^h$.
Esto es cierto, pues en caso contrario $a_i^h$ tendría grado 6 y es absurdo debido a que $a_i^h$ pertenece a $G''$.

Para resumir, hemos demostrado cuales son los nodos que pueden conformar un subgrafo cúbico $G''$ sobre el grafo $G'$, es decir,
hemos demostrado que si $G''$ es un subgrafo cúbico de $G'$ entonces en $G''$ solo existe exactamente un ciclo entre $C_i^1,C_i^2,C_i^3$
y que si un \textit{nodo arista} $a_i^h$ asociado a la arista $e_i=<v_x,v_y>$ está en $G''$ entonces, en $G''$ solo puede existir exactamente
uno de los ciclos $C_x^h$ y $C_y^h$. Todo esto junto demuestra que si $G''$ es un subgrafo cúbico, entonces el colocar el nodo $v_i$,
representante del ciclo $C_i^h$ en $G''$ en la partición $V^h$, con $1\leq h\leq 3$, crea una 3-coloración valida en $G$. Luego queda demostrado
que el problema \ref{prob_1} es un problema \textbf{NP-completo}.


\section{Solución}

Debido a que demostramos que el problema \ref{prob_1} es \textbf{NP-completo}, se sabe que actualmente no se conocen soluciones correctas a estos problemas y
que a su vez se ejecuten en tiempo polinomial (pues en caso contrario se tendría que \textbf{P=NP}).

Es sabido que para resolver problemas \textbf{NP-completos} hay que sacrificar correctitud o velocidad. Es por esto que probaremos 2 soluciones distintas en esta sección.
Una solución buscará la correctitud sacrificando velocidad, este es el caso de la solución por Fuerza Bruta. La otra solución que expondremos es basada en el area de metaheuristicas.
Para esta última solución transformaremos el problema de decisión \ref{prob_1} en un problema de optimización cuya solución óptima nos permita
encontrar la solución en el problema original.
% Para el problema \ref{prob_1} implementaremos 2 soluciones. Una será la implementación por Fuerza Bruta, que aunque es correcta también es exponencial. La otra será utilizando
% un algoritmo de metaheuristica sobre una versión modificada del problema \ref{prob_1} (en la que este se transforma en un problema de optimización) que aunque no da la solución optima, se sabe que encuentra soluciones
% cerca de la optimalidad en mucho menos tiempo que las opciones por Fuerza Bruta.

\subsection{Fuerza Bruta}

Dado un grafo $G=(V,E)$ la solución por Fuerza Bruta sigue construye todos los posibles subgrafos de $G$ utilizando Backtrack con un enfoque bottom-up, es decir,
inicialmente se crea un grafo $G'$ con todos los nodos de $G$ pero sin aristas. Se prosigue con todas las formas de elegir la primera arista, luego la segunda arista y así
sucesivamente. A la hora de agregar una nueva arista a $G'$ se analiza si algún nodo de este llega a obtener un grado mayor que 3, en cuyo caso no se agrega la arista y se poda
la recursividad por esa rama. Si en algún punto del algoritmo se cumple que el grafo $G'$ tiene todos los nodos de grado 0 o 3, se devuelve el grafo y se da una respuesta afirmativa.
Si al explorar todos los posibles subgrafos, no se encontró la solución, entonces se devuelve una respuesta negativa al problema.

Como potencialmente este algoritmo explora todas las posibles combinaciones de aristas de $G = (V,E)$, donde $m=|E|$, entonces su complejidad es $O(2^m)$


% \subsection{}

\bigskip
%%%%%%%%%%%%%%%
\begin{thebibliography}{20}
    \bibitem{3-color} R. M. Karp: \emph{Reducibility among combinatorial problems.} R. E. Miller et al. (eds.), Complexity of Computer Computations (1972), pp 85-103.
\end{thebibliography}

\end{document}
